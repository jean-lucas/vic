% Document uses 12 pt font
% 1 in margins
% Contains a relative path for images

\documentclass [11pt]{article}

% page geometry 
\usepackage[margin=1in]{geometry}


% ----------  PACKAGES START ------------ %

% Table cell color
\usepackage[table]{xcolor}


% VIC title package
\usepackage{cabin}
\usepackage[T1]{fontenc}

% default font package
%\usepackage{times}
\usepackage{helvet}
\renewcommand{\familydefault}{\sfdefault}

% ---------- End Font Packages -------------- %

% Title Packages
\usepackage{titlesec}
\usepackage{titletoc}

% Image Package
\usepackage{graphicx}

% Table Packages
\usepackage{longtable}
\usepackage{multirow}
\usepackage{multicol}
\usepackage{multirow}
\usepackage{array}
\renewcommand{\arraystretch}{1.4}% Spread rows out evenly in table

% Color Packages
\usepackage{color}   
\definecolor{sectionC}{rgb}{0.016,0.227,.365}
\definecolor{subsectionC}{rgb}{.87,0.87,.87}
\definecolor{subsubsectionC}{rgb}{.94,.93,.90}
\definecolor{tableCell}{rgb}{.96,.95,.90}


% List package
\usepackage{enumitem}
\setenumerate{itemsep=0pt, itemindent=0in,leftmargin=0.5in}

% Paragraph parameter

\setlength{\parindent}{0pt}


% ------------- Creates a linked Table of Contents  Start --------------- %
\usepackage{hyperref}
\hypersetup{
colorlinks=true, %set true if you want colored links
linktoc=all,     %set to all if you want both sections and subsections linked
linkcolor=black,}  %choose some color if you want links to stand out

% ------------- Creates a click-able Table of Contents  End--------------- %

% ---------- PACKAGES END ------------ %



% ------------------- START HEADER AND FOOTER ---------------------------%
\usepackage{fancyhdr}

% Helps with the n of total n pages
\usepackage{lastpage}

\pagestyle{fancy}

% Header
\lhead{Draft System Requirements }
\rhead{Revision: 0}
\fancyhead[LE,CO]{VIC - Group 6}

% Removes line under the header 
\renewcommand{\headrulewidth}{0pt}
\setlength{\headsep}{.2in}

% Footer 

% Set the right side of the footer to be the page number
\fancyfoot[R]{Page \textbf{\thepage}\ of \textbf{\pageref{LastPage}}}
\fancyfoot[C]{}

% ------------------- END HEADER AND FOOTER ---------------------------%




% -------- SECTION AND SUBSECTION FORMATING START -------- % 
% starts the 
%\setcounter{section}{1}


\titleformat{\section} % Section
{\normalfont \fontsize{14}{14} \bfseries}{}{0em}{\colorsection}

% Makes a background color
\newcommand{\colorsection}[1]{%
  \colorbox{sectionC}{\parbox{\dimexpr\textwidth-1\fboxsep}{\color{white}\Large\thesection\ \hspace{1mm} #1}}}

% Makes a background color
\titleformat{\subsection} % Subsection
{\normalfont \fontsize{12}{12}  \bfseries}{}{0em}{\colorsubsection }

\newcommand{\colorsubsection}[1]{%
  \colorbox{subsectionC}{\parbox{\dimexpr \textwidth -1\fboxsep}{\large\thesubsection\ #1}}}


% Makes a background color
\titleformat{\subsubsection} % Subsubsection
{\normalfont \fontsize{12}{12} \bfseries}{}{0em}{\colorsubsubsection}

\newcommand{\colorsubsubsection}[1]{%
  \colorbox{subsubsectionC}{\parbox{\dimexpr\textwidth-1\fboxsep}{\thesubsubsection\ #1}}}

% -------- SECTION AND SUBSECTION FORMATING END -------- % 
\usepackage{lipsum}


% -----  IMAGE PATH START -----%
% Relative Image Path
\graphicspath {figures/}
% -----  IMAGE PATH END -----%

% ------ PARAGRAPH FORMAT START ----%
%\setlength{\parskip}{.2em}% Sets the space between new paragraph items 
\setlength{\parindent}{0em} % paragraph indent
% ------ PARAGRAPH FORMAT END ----%




%------------------------------TOC FORMAT START --------------------------------%
\usepackage{tocloft}

% Section indentations
\cftsetindents{section}{0em}{1.5em}
\cftsetindents{subsection}{1em}{2em}
\cftsetindents{subsubsection}{2em}{3em}

% Toc title size
\renewcommand\cfttoctitlefont{\Large\bfseries}

% Removes bold headings from toc
%\renewcommand{\cftsecfont}{\normalfont}

% Removes bold heading page numbers from toc
\renewcommand{\cftsecpagefont}{\normalfont}

% add dots after headings
%\renewcommand{\cftsecleader}{\cftdotfill{\cftdotsep}} 


% number of section headings we want to see in toc
\setcounter{tocdepth}{2}

% Spaceing before headings in toc
\setlength{\cftbeforesecskip}{6pt}

% ------------------------------TOC FORMAT END --------------------------------%








% -------------- DOCUMENT START ---------------%
\begin{document}

% --------- TITLE PAGE START ------- %
\begin {center} 

\thispagestyle{empty}
\vspace*{4.5cm}

% Logo Insertion
\begin {figure}[h!]
\centering
\hspace{-10mm}\includegraphics [scale = .5, trim={.4cm 0 .8cm 0},clip] {figures/vic_logo.png}
\end {figure}

{\fontfamily{\cabinfamily}\selectfont
\Huge{Vehicle Intersection Control} }

\vspace{1 cm}
{\Large{\textsc{McMaster University}}\\}  \vspace {1cm}
{\large Draft System Requirements\\ \vspace {0.4 cm} SE 4G06}  \vspace {1cm}

{\large \textsc{Group 6} \\} \vspace{1cm}

{
Alex Jackson \\
Jean Lucas Ferreira \\
Justin Kapinski\\
Mathew Hobers\\
Radhika Sharma\\
Zachary Bazen}




\end{center}

% --------- TITLE PAGE END------- %

\pagebreak

% Inserting table of contents and table of figures 

\tableofcontents
\listoftables
\listoffigures



\pagebreak

% -----------  REVISION HISTORY START ----------- %

%\section*{Revisions}
\thispagestyle{empty}
\section{Revisions}
\begin{longtable}{| p{.2\textwidth } | p{.23\textwidth } | p{.23\textwidth } | p{.23\textwidth } |}

\hline 
\centering \textbf{Date} & 
\multicolumn{1}{c}{\textbf {Revision Number}} &
\multicolumn{1}{|c}{\textbf {Authors}} & 
\multicolumn{1}{|c|}{\textbf {Comments}} \\ \hline

\multirow{4}{*}{\centering November 14, 2016}  & 
\multirow{4}{*}{Revision 0}& 
{Alex Jackson \newline
Jean Lucas Ferreira \newline
Justin Kapinski\newline
Mathew Hobers\newline
Radhika Sharma\newline
Zachary Bazen}
&
 \multirow{4}{*}{N/A} \\ 
\hline 

\caption{VIC Table of Revisions} 
\end{longtable}
% -----------  REVISION HISTORY END ----------- %
\pagebreak

%---------------------------- PROJECT DRIVERS ------------------------%
% heading in document


\section {\textbf{Project Drivers}}


\subsection{The Purpose of the Project} 
The purpose of this project is to create a system that allows autonomous cars to navigate through  intersections. This will be accomplished by providing an appropriate order for the vehicles to proceed through the intersection. When multiple autonomous cars arrive at an intersection simultaneously, due to the lack of a decision making protocol, the cars have no way of determining in which order to proceed. \newline


Vehicle Intersection Control (also known as VIC) will allow autonomous vehicles to make navigation decisions at intersections. In addition, VIC will be able to dynamically handle changing scenarios at an intersection without running into deadlock or stalemate situations. To ensure safety, VIC will allow cars to navigate through the intersection only after a unanimous consensus has been made. \newline

The following document will outline the functional and nonfunctional requirements of VIC.  Other topics that will be covered pertaining to VIC will include: Scope, Project Drivers, Project Constraints, and Project Issues.

\subsection{The Client, the Customer, and Other Stakeholders}

\subsubsection{Client and Customer}
	The client and the customer for this project is Shaun Marshall, the engineering group manager at General Motors


\subsubsection{Stakeholders}
 
 	The stakeholders consists of:
 		\begin{itemize}

 		\item The developers and system designers of VIC
 		\item Dr. Alan Wassyng, the project supervisor
 		\item The teaching assistans of the courses, as the evaluators of the overall system.
 		\end{itemize} 

\subsection{Users of the Product} 

This product is expected to be used by researchers in the field of autonomous vehicles.



%---------------------------- PROJECT CONSTRAINTS ------------------------%
% heading in document
\section{\textbf{Project Constraints}}


% Mandated Constraints
\subsection{Mandated Constraints}
Vehicle intersection control has several mandated constraints tabled below. 
\begin{longtable}{| p{.2\textwidth } | p{.75\textwidth } | }\hline 
\textbf{MC1} & \textbf{Remote control cars must be 1/10 scale} \\ \hline
\textbf{RMC1} & The remote control cars must fit the requirements of an existing track that was created for previous capstone projects\\ \hline 

\end{longtable}

\begin{longtable}{| p{.2\textwidth } | p{.75\textwidth } | }\hline 

\textbf{MC2}& \textbf{Remote control cars must be electric}\\ \hline 
\textbf{RMC2} & Operating conditions are indoors  \\ \hline 

\end{longtable}

\begin{longtable}{| p{.2\textwidth } | p{.75\textwidth } | }\hline 
\textbf{MC3} & \textbf{The cost of the project must not exceed \$700 dollars} \\ \hline
\textbf{RMC3} & This is to ensure an off-the-shelf solution can not be purchased. It also ensures the project remains economically feasible \\ \hline
\end{longtable}


\subsection{Naming Conventions and Definitions}
% Naming Conventions
\subsubsection{Naming Conventions}
\begin{longtable}{ |p{.13\textwidth }  p{.825\textwidth }|}  \hline
\textbf{T\#} &  Track requirement identification  and number \\ 

\cellcolor{tableCell}\textbf{V\#}  & \cellcolor{tableCell}Remote control vehicle requirement identification  and number \\ 

\textbf{IC\#} & Intersection control requirement identification  and number \\ 

\cellcolor{tableCell}\textbf{MC\#} &  \cellcolor{tableCell}Mandated project constraints identification and number \\ 

\textbf{RMC\#} & Rational for mandated project constraints identification and number\\ 

\cellcolor{tableCell}\textbf{A\#} & \cellcolor{tableCell}Project assumptions identification and number \\ 

\textbf{RA\#} & Rational for project assumptions identification and number \\ 

\cellcolor{tableCell}\textbf{VIC} & \cellcolor{tableCell}Vehicle intersection control \\ \hline


\end{longtable}
%
%\begin{enumerate}
%	\itemsep0pt 
% 	\item \textbf{T\#} - Track requirement identification  and number
%	\item \textbf{V\#} - Remote control vehicle requirement identification  and number				
%	\item \textbf{IC\#} - Intersection control requirement identification  and number
%	\item \textbf{MC\#} - Mandated project constraints identification and number 
%	\item \textbf{RMC\#} - Rational for mandated project constraints identification and number
%	\item \textbf{A\#} - Project assumptions identification and number
%	\item \textbf{RA\#} - Rational for project assumptions identification and number
%	\item \textbf{VIC} - Vehicle intersection control
%\end{enumerate}

% Definitions
\subsubsection{Definitions}
\begin{enumerate}
	\itemsep0pt
	\item \textbf{VIC} - The name given to the overall intersection control system
\end{enumerate}

	
\subsection{Relevant Facts and Assumptions} 

\subsubsection{Relevant Facts}
\begin{itemize}
	\item N/A
\end{itemize}

\subsubsection{Assumptions}
VIC assumptions tabled below. 
\begin{longtable}{| p{.2\textwidth } | p{.75\textwidth } | }\hline 
\textbf{A1} & \textbf{Ideal driving conditions on the track} \\ \hline
\textbf{RA1} & Track is situated indoors \\ \hline 
\end{longtable}

\begin{longtable}{| p{.2\textwidth } | p{.75\textwidth } | }\hline 
\textbf{A2} & \textbf{Intersection is a four way stop} \\ \hline
\textbf{RA2} &  Different intersection arrangements beyond the scope of this project \\ \hline
\end{longtable}

\begin{longtable}{| p{.2\textwidth } | p{.75\textwidth } | }\hline 
\textbf{A3} & \textbf{Only autonmous car will be present on the track} \\ \hline
\textbf{RA3} &  This will help simplify the scope of the project\\ \hline
\end{longtable}

\begin{longtable}{| p{.2\textwidth } | p{.75\textwidth } | }\hline 
\textbf{A4} & \textbf{Cars will not have a large variance in size} \\ \hline
\textbf{RA4} &  The 1/10th model cars will only consists of sedan or coupe styled cars. We will not consider large vehicles such as trucks or buses.\\ \hline
\end{longtable}


\section{Context Diagrams}
\begin{figure} [h!]
	\centering
	\includegraphics [scale = 0.7] {figures/IC_ContextDiagram.pdf}
	\caption{Intersection Controller Context Diagram}
\end{figure}
\begin{figure} [h!]
	\centering
	\includegraphics [scale =0.7] {figures/CarCtrl_ContextDiag.pdf}
	\caption{Car Controller Context Diagram}
\end{figure}

\section{Constants}
\begin{itemize}
\item TBD

\end{itemize}


\section{Monitored and Controlled Variables}
\begin{itemize}
\item TBD

\end{itemize}

%---------------------------- Functional Requirements ------------------------%

\section {Functional Requirements} 
The requirements for this project are separated into the three main components of the system: the track, vehicle, and intersection controller.



% Track Requirements

% if one requirements gets deleted it will be shown in the requirement likelihood of change as ??
% with all the other labels numbers updated

 % ------ NOTE ------% 
% CAN ONLY ADD REQUIREMENTS TO END OF LIST WITH OWN UNIQUE LABEL
% Which would then have to be put at the end of the requirement group in likelihood of change section
% If we add requirements in the middle, the likelihood of change will get messed up

\subsection{Track Functional Requirements}
\begin{enumerate}[label=\textbf{T\arabic*:}, ref=T\arabic*, leftmargin =0.8in]
	\item \label{T1} \hyperref[sec:changeL]{The track must have lanes } % Unlikely
	\item \label{T2} \hyperref[sec:changeL]{The track must have an intersection} % Unlikely
	\item \label {T3} \hyperref[sec:changeL]{The track must have an object to indicate stopping at an intersection}  % Unlikely
\end{enumerate}





% Vehicle Requirements
\subsection{Vehicle Functional Requirements}

\begin{enumerate}[label=\textbf{V\arabic*:}, ref =V\arabic*, leftmargin=0.7in]

	
	\item \label{V1}  \hyperref[sec:changeL]{The vehicle must be able to send and receive signals to and from the system} infrastructure % Llikely
	
	\item \label{V2} \hyperref[sec:changeL]{The vehicle must be able to detect lanes and follow them} % Unlikely
    
    \item \label{V3} \hyperref[sec:changeL]{The vehicle must be able to detect intersections}  % Unlikely
    
    \item \label{V4} \hyperref[sec:changeL]{The vehicle must be able to stop at intersections} % Unlikely
    
    \item \label{V5} \hyperref[sec:changeL]{The vehicle must be able to navigate through intersections} % Unlikely
    
    \item \label{V6} \hyperref[sec:changeL]{The vehicle must be able to avoid obstacles} % Unlikely
    
    \item \label{V7} \hyperref[sec:changeL]{The vehicle must follow the laws of the Highway Traffic act} % Unlikely
\end{enumerate}





% VI Controller Requirements

\subsection{Intersection Controller Functional Requirements}
\begin{enumerate}[label=\textbf{IC\arabic*:}, ref =IC\arabic*, leftmargin=0.8in]
%	\setlength{\itemindent}{.5in}

	\item \label{IC1} \hyperref[sec:changeL]{The system infrastructure must be able to detect if there is a car at the intersection } % Likely

	
	\item \label{IC2}\hyperref[sec:changeL]{The system infrastructure must be able to differentiate between autonomous and non autonomous cars} % Likely
	
	\item \label{IC3}\hyperref[sec:changeL]{The system infrastructure must be able to detect when a car has navigated through the intersection} % Likely

    \item \label{IC4}\hyperref[sec:changeL]{The system infrastructure must be able to determine the order in which the cars should proceed} % Likely
    
    \item \label{IC5} \hyperref[sec:changeL]{The system infrastructure must be able to signal to the vehicle when it is allowed to go through the intersection} % Likely
\end{enumerate}

\break
\section{Functional Decomposition Diagrams}
\begin{figure} [h!]
	\vspace*{-5mm}
	\centering
	\includegraphics [scale =.7] {figures/function_decomp_track_n.pdf}
	\caption{Functional Track Navigation Decomposition}
\end{figure}

\begin{figure} [h!]
\vspace*{-5mm}
	\centering
	\includegraphics [scale =.78] {figures/function_decomp_IC.pdf}
	\caption{Functional Intersection Controller Decomposition}
\end{figure}


\section{Functional Requirements Likelihood of Change} \label{sec:changeL}

\begin{longtable}{| p{.2\textwidth } | p{.3\textwidth } |  p{.3\textwidth } |}\hline 
\multicolumn{1}{|c|}{\textbf {Requirement Group}} & \multicolumn{1}{c|}{\textbf {Requirement}} & \multicolumn{1}{c|}{\textbf {Likelihood of Change}} \\ \hline

% Track Requirement
\multicolumn{1}{|c|}{\textbf {\multirow{3}{*}{\centering Track}}}  & \multicolumn{1}{c |}{\ref{T1}} & \multicolumn{1}{c |}{Unlikely} \\\cline{2-3}
                                 & \multicolumn{1}{c |}{\ref{T2}   } & \multicolumn{1}{c |}{Unlikely} \\\cline{2-3}
                                 & \multicolumn{1}{c |}{\ref{T3}} & \multicolumn{1}{c |}{Unlikely} \\\cline{2-3} \hline 


 
% Vehicle Requirement
\multicolumn{1}{|c|}{\textbf {\multirow{7}{*}{\centering Vehicle}}}  & \multicolumn{1}{c |}{\ref{V1}} & \multicolumn{1}{c |}{Likely} \\\cline{2-3}
                                 & \multicolumn{1}{c | }{\ref{V2}} & \multicolumn{1}{c |}{Unlikely} \\\cline{2-3}
                                 & \multicolumn{1}{c | }{\ref{V3}} & \multicolumn{1}{c |}{Unlikely} \\\cline{2-3}
                                 & \multicolumn{1}{c | }{\ref{V4}} & \multicolumn{1}{c |}{Unlikely} \\\cline{2-3}
                                 & \multicolumn{1}{c | }{\ref{V5}} & \multicolumn{1}{c |}{Unlikely} \\\cline{2-3}
                                 & \multicolumn{1}{c | }{\ref{V6}} & \multicolumn{1}{c |}{Unlikely} \\\cline{2-3}
                                 & \multicolumn{1}{c | }{\ref{V7}} & \multicolumn{1}{c |}{Unlikely} \\\cline{2-3} \hline 

% Intersection Requirement                               
\multicolumn{1}{|c|}{\textbf {\multirow{5}{*}{Intersection Control}}}  & \multicolumn{1}{c |}{\ref{IC1}} & \multicolumn{1}{c |}{Likely} \\\cline{2-3}
                                 & \multicolumn{1}{c | }{\ref{IC2}} & \multicolumn{1}{c |}{Likely} \\\cline{2-3}
                                 & \multicolumn{1}{c | }{\ref{IC3}} & \multicolumn{1}{c |}{Likely} \\\cline{2-3}
                                 & \multicolumn{1}{c | }{\ref{IC4}} & \multicolumn{1}{c |}{Likely} \\\cline{2-3}
                                 & \multicolumn{1}{c | }{\ref{IC5}} & \multicolumn{1}{c |}{Likely} \\\cline{2-3} \hline


\end{longtable}



%\section{The Scope of the Work}
%Insert Text Here.
%
%\section{The Scope of the Product} 
%Insert Text Here.
%
%\section{Functional Requirements and Data Requirements} 
%Insert Text Here.


%---------------------------- Nonfunctional Requirements ------------------------%

\section {Nonfunctional Requirements} 



\subsection {Look and Feel Requirements}
\subsubsection{Appearance Requirements}
	\begin{itemize}
		\item  N/A
	\end{itemize}
	% This is a test section


\subsubsection{Style Requirements}
	\begin{itemize}
		\item N/A
	\end{itemize}

\subsection{Usability and Humanity Requirements} 
\subsubsection{Ease of Use Requirements}
	\begin{itemize}
		\item N/A
	\end{itemize}

\subsubsection{Personalization and Internationalization Requirements}
	\begin{enumerate}[label=\textbf{\Alph*}:]
		\item The system must be able to function according to North American road standards
	\end{enumerate}

\subsubsection{Learning Requirements }
	\begin{itemize}
		\item N/A
	\end{itemize}

\subsubsection{Understandability and Politeness Requirements}
	\begin{itemize}
		\item N/A
	\end{itemize}
		
\subsubsection{Accessibility Requirements }
	\begin{itemize}
		\item N/A
	\end{itemize}
 
\subsection{Performance Requirements}
	Please note that the following  non functional requirements will be updated as the system is created and data is acquired.

\subsubsection{Speed Requirements }
	\begin{enumerate}[label=\textbf{\Alph*}:]
		\item The system must be able to determine an order and convey it to the vehicle before a soft deadline
	\end{enumerate}

\subsubsection{Safety-Critical Requirements }
	\begin{enumerate}[label=\textbf{\Alph*}:]
		\item The system must only signal a car to proceed when the intersection is clear
		\item The vehicle must stop within a safe distance of an obstacle
	\end{enumerate}	


\subsubsection{Precision Requirements}
	\begin{enumerate}[label=\textbf{\Alph*}:]
		\item The vehicle must not deviate from the lanes more than 1\%
	\end{enumerate}

\subsubsection{Reliability or Availability  Requirements}
	\begin{enumerate}[label=\textbf{\Alph*}:]
		\item The system must operate without failure 99\% of the time
		\item The vehicle system must operate as long as car's internal power supply is charged
	\end{enumerate}



\subsubsection{Robustness or Fault-Tolerance Requirements }
	\begin{enumerate}[label=\textbf{\Alph*}:]
		\item In the event of a complete vehicle system failure, the vehicle must come to a stop
	\end{enumerate}
	
\subsubsection{Capacity Requirements }
	\begin{enumerate}[label=\textbf{\Alph*}:]
		\item The intersection controller shall be able to manage one intersection at a time
		\item The intersection controller shall be able to communicate with a maximum of four cars at a time
	\end{enumerate}

\subsubsection{Scalability or Extensibility Requirements }
	\begin{itemize}
		\item N/A
	\end{itemize}
		
\subsubsection{Longevity Requirements }
	\begin{enumerate}[label=\textbf{\Alph*}:]
		\item Components should be functional for up to one year
	\end{enumerate}

% Operational and Environmental Requirements
\subsection{Operational and Environmental Requirements}
\subsubsection{Expected Physical Environment }
	\begin{enumerate}[label=\textbf{\Alph*}:]
		\item The track must be 1/10 scale of a real world intersection
	\end{enumerate}
		
\subsubsection{Requirements for Interacting with Adjacent Systems}
	\begin{enumerate}[label=\textbf{\Alph*}:]
		\item The components must be able to use the API of existing and partner components
	\end{enumerate}

\subsection{Maintainability and Support Requirements }
\subsubsection{Maintenance Requirements }
	\begin{enumerate}[label=\textbf{\Alph*}:]
		\item  Issues must be resolved within one week of discovering an error in the system
	\end{enumerate}

\subsubsection{Supportability Requirements }
	\begin{itemize}
		\item N/A
	\end{itemize}

\subsubsection{Adaptability Requirements}
	\begin{itemize}
		\item N/A
	\end{itemize}

\subsection{Security Requirements }
\subsubsection{Access Requirements }
	\begin{enumerate}[label=\textbf{\Alph*}:]
		\item All stated stakeholders have full access to the product
	\end{enumerate}

\subsubsection{Integrity Requirements }
	\begin{enumerate}[label=\textbf{\Alph*}:]
		\item The system will not be altered by external signals
	\end{enumerate}

\subsubsection{Privacy Requirements }
	\begin{itemize}
		\item N/A
	\end{itemize}

\subsubsection{Audit  Requirements }
	\begin{itemize}
		\item N/A
	\end{itemize} 

\subsubsection{Immunity Requirements  }
	\begin{itemize}
		\item N/A
	\end{itemize}

\subsection{Cultural and Political Requirements } 
\subsubsection{Cultural Requirements }
	\begin{itemize}
		\item N/A
	\end{itemize}

\subsubsection{Political Requirements }
	\begin{itemize}
		\item N/A
	\end{itemize}


\subsection{Legal Requirements}
\subsubsection{Compliance Requirements }
	\begin{itemize}
		\item N/A
	\end{itemize}
	
\subsubsection{Standards Requirements }
	\begin{itemize}
		\item N/A
	\end{itemize}


%---------------------------- Project Issues ------------------------%

\section {Project Issues} 


\subsection{Open Issues}
	\begin{enumerate}[label=\textbf{\Alph*}:]
		\item The track design is unknown
	\end{enumerate}

\subsection{Off-the-Shelf Solutions}

\subsubsection{Ready-Made Products}
	\begin{enumerate}[label=\textbf{\Alph*}:]
		\item Autonomous Intersection Management, and existing product that partially solves our problem
	\end{enumerate}


\subsection{Risks}
	\begin{enumerate}[label=\textbf{\Alph*}:]
		\item Component failure
		\item Parts damaged
		\item Potential to minor injuries to humans
	\end{enumerate}
	

	
\subsection{Costs}	
		
		The general budget for the major components are as follow:

		\begin{enumerate}[label=\textbf{\Alph*}:]
			\item 1/10th model car \$200.0 each
			\item Cameras and sensors \$100.00
			\item Micro-controllers \$200  
		\end{enumerate}


\subsection{Waiting Room}
	\begin{enumerate}[label=\textbf{\Alph*}:]
		\item Having the system work with other autonomous car models
		\item Having the system work with non-autonomous cars
	\end{enumerate} 



\end{document}
