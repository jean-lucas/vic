% Document uses 12 pt font
% 1 in margins
% Contains a relative path for images

\documentclass [12pt]{article}

% page geometry 
\usepackage[margin=1in]{geometry}


% ----------  PACKAGES START ------------ %

% VIC title package
\usepackage{cabin}
\usepackage[T1]{fontenc}

% default font package
%\usepackage{times}
\usepackage{helvet}
\renewcommand{\familydefault}{\sfdefault}

% ---------- End Font Packages -------------- %

% Title Packages
\usepackage{titlesec}
\usepackage{titletoc}

% Image Package
\usepackage{graphicx}

% Table Packages
\usepackage{longtable}
\usepackage{multirow}
\usepackage{multicol}
\usepackage{multirow}
\usepackage{array}
\renewcommand{\arraystretch}{1.4}% Spread rows out evenly in table

% Color Packages
\usepackage{color}   
\definecolor{sectionC}{rgb}{0.016,0.227,.365}
%\definecolor{sectionC}{rgb}{1,0.749,.40}
%\definecolor{sectionT}{rgb}{0.173,0.169,.165}
%\definecolor{subsectionC}{rgb}{0.898,0.792,.718}
\definecolor{subsectionC}{rgb}{.87,0.87,.87}
\definecolor{subsubsectionC}{rgb}{.94,.93,.90}


% List package
\usepackage{enumerate}


% ------------- Creates a linked Table of Contents  Start --------------- %
\usepackage{hyperref}
\hypersetup{
colorlinks=true, %set true if you want colored links
linktoc=all,     %set to all if you want both sections and subsections linked
linkcolor=black,}  %choose some color if you want links to stand out

% ------------- Creates a click-able Table of Contents  End--------------- %

% ---------- PACKAGES END ------------ %



% ------------------- START HEADER AND FOOTER ---------------------------%
\usepackage{fancyhdr}

% Helps with the n of total n pages
\usepackage{lastpage}

\pagestyle{fancy}

% Header
\lhead{Draft System Requirements }
\rhead{Revision: 0}
\fancyhead[LE,CO]{VIC - Group 6}

% Removes line under the header 
\renewcommand{\headrulewidth}{0pt}
\setlength{\headsep}{.2in}

% Footer 

% Set the right side of the footer to be the page number
\fancyfoot[R]{Page \textbf{\thepage}\ of \textbf{\pageref{LastPage}}}
\fancyfoot[C]{}

% ------------------- END HEADER AND FOOTER ---------------------------%




% -------- SECTION AND SUBSECTION FORMATING START -------- % 
% starts the 
%\setcounter{section}{1}


\titleformat{\section} % Section
{\normalfont \fontsize{14}{14} \bfseries}{}{0em}{\colorsection}

% Makes a background color
\newcommand{\colorsection}[1]{%
  \colorbox{sectionC}{\parbox{\dimexpr\textwidth-1\fboxsep}{\color{white}\Large\thesection\ \hspace{1mm} #1}}}

% Makes a background color
\titleformat{\subsection} % Subsection
{\normalfont \fontsize{12}{12}  \bfseries}{}{0em}{\colorsubsection }

\newcommand{\colorsubsection}[1]{%
  \colorbox{subsectionC}{\parbox{\dimexpr\textwidth-1\fboxsep}{ \thesubsection\ #1}}}


% Makes a background color
\titleformat{\subsubsection} % Subsubsection
{\normalfont \fontsize{12}{12} \bfseries}{}{0em}{\colorsubsubsection}

\newcommand{\colorsubsubsection}[1]{%
  \colorbox{subsubsectionC}{\parbox{\dimexpr\textwidth-1\fboxsep}{\thesubsubsection\ #1}}}

% -------- SECTION AND SUBSECTION FORMATING END -------- % 
\usepackage{lipsum}


% -----  IMAGE PATH START -----%
% Relative Image Path
\graphicspath {figures/}
% -----  IMAGE PATH END -----%

% ------ PARAGRAPH FORMAT START ----%
%\setlength{\parskip}{.2em}% Sets the space between new paragraph items 
\setlength{\parindent}{0em} % paragraph indent
% ------ PARAGRAPH FORMAT END ----%




%------------------------------TOC FORMAT START --------------------------------%
\usepackage{tocloft}

% Section indentations
\cftsetindents{section}{0em}{1.5em}
\cftsetindents{subsection}{1em}{2em}
\cftsetindents{subsubsection}{2em}{3em}

% Toc title size
\renewcommand\cfttoctitlefont{\Large\bfseries}

% Removes bold headings from toc
%\renewcommand{\cftsecfont}{\normalfont}

% Removes bold heading page numbers from toc
\renewcommand{\cftsecpagefont}{\normalfont}

% add dots after headings
%\renewcommand{\cftsecleader}{\cftdotfill{\cftdotsep}} 


% number of section headings we want to see in toc
\setcounter{tocdepth}{2}

% Spaceing before headings in toc
\setlength{\cftbeforesecskip}{6pt}

% ------------------------------TOC FORMAT END --------------------------------%








% -------------- DOCUMENT START ---------------%
\begin{document}

% --------- TITLE PAGE START ------- %
\begin {center} 

\thispagestyle{empty}
\vspace*{4.5cm}

% Logo Insertion
\begin {figure}[h!]
\centering
\hspace{-10mm}\includegraphics [scale = .5, trim={.4cm 0 .8cm 0},clip] {figures/vic_logo.png}
\end {figure}

{\fontfamily{\cabinfamily}\selectfont
\Huge{Vehicle Intersection Control} }

\vspace{1 cm}
{\Large{\textsc{McMaster University}}\\}  \vspace {1cm}
{\large Draft System Requirements\\ \vspace {0.4 cm} SE 4G06}  \vspace {1cm}

{\large \textsc{Group 6} \\} \vspace{1cm}

{
Alex Jackson \\
Jean Lucas Ferreira \\
Justin Kapinski\\
Matthew Hobers\\
Radhika Sharma\\
Zachary Bazen}




\end{center}

% --------- TITLE PAGE END------- %

\pagebreak

% Inserting table of contents and table of figures 

\tableofcontents
\listoftables

\pagebreak

% -----------  REVISION HISTORY START ----------- %

%\section*{Revisions}
\thispagestyle{empty}
\noindent {\textbf{\large Revisions}}
\addcontentsline{toc}{section}{\textbf{\normalsize Revisions}}
\begin{longtable}{| p{.2\textwidth } | p{.23\textwidth } | p{.23\textwidth } | p{.23\textwidth } |}

\hline 
\centering \textbf{Date} & 
\multicolumn{1}{c}{\textbf {Revision Number}} &
\multicolumn{1}{|c}{\textbf {Authors}} & 
\multicolumn{1}{|c|}{\textbf {Comments}} \\ \hline

\multirow{4}{*}{\centering November 7, 2016}  & 
\multirow{4}{*}{Revision 0}& 
{Alex Jackson \newline
Jean Lucas Ferreira \newline
Justin Kapinski\newline
Matthew Hobers\newline
Radhika Sharma\newline
Zachary Bazen}
&
 \multirow{4}{*}{N/A} \\ 
\hline 

\caption{VIC Table of Revisions} 
\end{longtable}
% -----------  REVISION HISTORY END ----------- %
\pagebreak

%---------------------------- PROJECT DRIVERS ------------------------%
% heading in document

\noindent{\textbf{\Large*** These headings to be double checked against template to ensure no topics missed ***}}

\section {\textbf{Project Drivers}}


\subsection{The Purpose of the Project} 
The purpose of this project is to allow autonomous cars to identify and navigate intersections by providing an appropriate order to proceed through the intersection. Currently multiple autonomous cars arriving at an intersection a the same time have no decision protocol that will determine which car proceeds first through the intersection.  \textbf{\{((From project goals)}
VIC will allow autonomous vehicles to identify intersections and form a unanimous consensus
of the order in which vehicles should proceed through an intersection. In addition, VIC will be
able to dynamically handle changing scenarios at an intersection without running into deadlock
or stalemate situations. To ensure safety, VIC will allow cars to navigate through the intersection
only after a unanimous consensus has been made. \textbf{\}}

\subsection{The Client, the Customer, and Other Stakeholders}

\subsubsection{Client}
\lipsum[1]  % generating text, take this out

\subsubsection{Customer}
 Insert Text Here.

\subsubsection{Stackholders}
 Insert Text Here.

\subsection{Users of the Product} 
Insert Text Here.\\


%---------------------------- PROJECT CONSTRAINTS ------------------------%
% heading in document
\section{\textbf{Project Constraints}}


\subsection{Mandated Constraints}
\lipsum[1]  % generating text, take this out
\subsection{Naming Conventions and Definitions}
	Insert Text Here.
	
\subsection{Relevant Facts and Assumptions} 
	Insert Text Here.


%---------------------------- Functional Requirements ------------------------%

\section {Functional Requirements} 
This section is taken from IEEE


% Track Requirements
\begin{enumerate}[\textbf{TRK-1:}]
	\setlength{\itemindent}{.2in}
	\itemsep0pt 
	\item Track Requirement 1
	\item Track Requirement 2

\end{enumerate}


% Vehicle Requirements
\begin{enumerate}[\textbf{VHL-1:}]
	\setlength{\itemindent}{.2in}
	\itemsep0pt 
	\item Requirement 1
	\item Requirement 2

\end{enumerate}


% VI Controller Requirements
\begin{enumerate}[\textbf{ITC-1:}]
	\setlength{\itemindent}{.2in}
	\itemsep0pt 
	\item Controller Requirement 1
	\item Controller Requirement 2

\end{enumerate}




%\section{The Scope of the Work}
%Insert Text Here.
%
%\section{The Scope of the Product} 
%Insert Text Here.
%
%\section{Functional Requirements and Data Requirements} 
%Insert Text Here.


%---------------------------- Nonfunctional Requirements ------------------------%

\section {Nonfunctional Requirements} 



\subsection {Look and Feel Requirements}
\subsubsection{Appearance Requirements}
Insert Text Here. 

\subsubsection{Style Requirements}
Insert Text Here. 


\subsection{Usability and Humanity Requirements} 
\subsubsection{Ease of Use Requirements}
Insert Text Here.

\subsubsection{Personalization and Internationalization Requirements}
Insert Text Here.

\subsubsection{Learning Requirements }
Insert Text Here.

\subsubsection{Understandability and Politeness Requirements}
Insert Text Here.
		
\subsubsection{Accessibility Requirements }
Insert Text Here.
 
\subsection{Performance Requirements}
\subsubsection{Speed Requirements }
Insert Text Here. 

\subsubsection{Safety-Critical Requirements }
		Insert Text Here. 	

\subsection{Precision Requirements}
		Insert Text Here.

\subsubsection{Reliability or Availability  Requirements}
		Insert Text Here.

\subsubsection{Robustness or Fault-Tolerance Requirements }

		Insert Text Here.
\subsubsection{Capacity Requirements }
		Insert Text Here.

\subsubsection{Scalability or Extensibility Requirements }
		Insert Text Here. 
		
\subsubsection{Longevity Requirements }
		Insert Text Here.

\subsection{Operational an Environmental Requirements}
\subsubsection{Expected Physical Environment }
		Insert Text Here.
		
\subsubsection{Requirements for Interacting with Adjacent Systems}
		Insert Text Here.

\subsubsection{Production Requirements}
		Insert Text Here. 

\subsubsection{Release Requirements}
		Insert Text Here.		

\subsection{Maintainability and Support Requirements }
\subsubsection{Maintenance Requirements }
		Insert Text Here.

\subsubsection{Supportability Requirements }
		Insert Text Here.

\subsubsection{Adaptability Requirements}
		Insert Text Here.


\subsection{Security Requirements }
\subsubsection{Access Requirements }
		Insert Text Here.

\subsubsection{Integrity Requirements }
		Insert Text Here.

\subsubsection{Privacy Requirements }
		Insert Text Here.

\subsubsection{Audit  Requirements }
		Insert Text Here. 

\subsubsection{Immunity Requirements  }
		Insert Text Here.

\subsection{Cultural and Political Requirements } 
\subsubsection{Cultural Requirements }
		Insert Text Here.

\subsubsection{Political Requirements }
		Insert Text Here.


\subsection{Legal Requirements}
\subsubsection{Compliance Requirements }
		Insert Text Here.
\subsubsection{Standards Requirements }
		Insert Text Here.




%---------------------------- Project Issues ------------------------%

\section {Project Issues} 


\subsection{Open Issues}
	Insert Text Here.

\subsection{Off-the-Shelf Solutions}
	Insert Text Here.

\subsection{New Problems}
	Insert Text Here.

\subsection{Migration to the New Product} 
	Insert Text Here.

\subsection{Risks}
	Insert Text Here.
	
\subsection{Costs}	
	Insert Text Here.

\subsection{User Documentation and Training}
	Insert Text Here.

\subsection{Waiting Room}
	Insert Text Here. 






\end{document}







%		% Requirement 1
%		\begin{longtable}{|p{.83\textwidth } |}\hline
%
%		
%		\begin{tabular}{m{3.4cm}  m{4.8cm} m{4.7cm} }
%			Requirement \# \textbf {1} & Requirement Type: \textbf {N/A} & Event /use case \#: \textbf {N/A} \\\newline
%		\end {tabular}
%
%			\textbf {Description:} Insert Text Here.\\ \newline
%			\textbf {Rational: } Insert Text Here.\\\newline 
%			\textbf {Originator:} Insert Text Here.\\ \newline
%			\textbf {Fit Criterion: }Insert Text Here.  \\ \newline 
%
%			\begin{tabular}{m{6cm}  m{8cm}}
%			Customer Satisfaction: \textbf {} & Customer Dissatisfaction: \textbf {} \\\newline
%			Priority: & Conflicts: \\\newline
%			\end {tabular}		
%
%			Supporting Materials: \\ \newline
%			History: \textbf {Date}\\ \hline
%		\end {longtable}
%		