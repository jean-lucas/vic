% Document uses 12 pt font
% 1 in margins
% Contains a relative path for images

\documentclass [10pt]{article}

% page geometry 
\usepackage[margin=1in]{geometry}


% ----------  PACKAGES START ------------ %

% Table cell color package and highlighting
\usepackage[table]{xcolor}
\usepackage{color,soul}

% VIC title package
\usepackage{cabin}
\usepackage[T1]{fontenc}

% default font package
%\usepackage{times}
\usepackage{helvet}
%\renewcommand{\familydefault}{\sfdefault}

% ---------- End Font Packages -------------- %

% Title Packages
\usepackage{titlesec}
\usepackage{titletoc}

% Image Package
\usepackage{graphicx}

% Table Packages
\usepackage{longtable}
\usepackage{multirow}
\usepackage{multicol}
\usepackage{multirow}
\usepackage{array}
\renewcommand{\arraystretch}{1.4}% Spread rows out evenly in table
\setlength{\LTpre}{0.5pt} % Reduces white space around tables (top)
%\setlength{\LTpost}{0pt} % Reduces white space around tables (bottom)

% Color Packages
\usepackage{color}   
\definecolor{sectionC}{rgb}{0.016,0.227,.365}
\definecolor{subsectionC}{rgb}{.87,0.87,.87}
\definecolor{subsubsectionC}{rgb}{.94,.93,.90}
\definecolor{tableCell}{rgb}{.96,.95,.90}


% List package
\usepackage{enumitem}
\setenumerate{itemsep=0pt, itemindent=0in,leftmargin=0.5in}


% Paragraph parameter

\setlength{\parindent}{0pt}


% ------------- Creates a linked Table of Contents  Start --------------- %
\usepackage{hyperref}
\hypersetup{
colorlinks=true, %set true if you want colored links
linktoc=all,     %set to all if you want both sections and subsections linked
linkcolor=black,}  %choose some color if you want links to stand out

% ------------- Creates a click-able Table of Contents  End--------------- %

% ---------- PACKAGES END ------------ %








% -------- SECTION AND SUBSECTION FORMATING START -------- % 
% starts the 
%\setcounter{section}{1}


\titleformat{\section} % Section
{\normalfont \fontsize{14}{14} \bfseries}{}{0em}{\colorsection}

% Makes a background color
\newcommand{\colorsection}[1]{%
  \colorbox{sectionC}{\parbox{\dimexpr\textwidth-1\fboxsep}{\color{white}\Large\thesection\ \hspace{1mm} #1}}}

% Makes a background color
\titleformat{\subsection} % Subsection
{\normalfont \fontsize{12}{12}  \bfseries}{}{0em}{\colorsubsection }

\newcommand{\colorsubsection}[1]{%
  \colorbox{subsectionC}{\parbox{\dimexpr \textwidth -1\fboxsep}{\large\thesubsection\ #1}}}


% Makes a background color
\titleformat{\subsubsection} % Subsubsection
{\normalfont \fontsize{12}{12} \bfseries}{}{0em}{\colorsubsubsection}

\newcommand{\colorsubsubsection}[1]{%
  \colorbox{subsubsectionC}{\parbox{\dimexpr\textwidth-1\fboxsep}{\thesubsubsection\ #1}}}

% -------- SECTION AND SUBSECTION FORMATING END -------- % 
\usepackage{lipsum}


% -----  IMAGE PATH START -----%
% Relative Image Path
\graphicspath {figures/}
% -----  IMAGE PATH END -----%

% ------ PARAGRAPH FORMAT START ----%
%\setlength{\parskip}{.2em}% Sets the space between new paragraph items 
\setlength{\parindent}{0em} % paragraph indent
% ------ PARAGRAPH FORMAT END ----%




%------------------------------TOC FORMAT START --------------------------------%
\usepackage{tocloft}



% Section indentations
\cftsetindents{section}{0em}{1.5em}
\cftsetindents{subsection}{1em}{2em}
\cftsetindents{subsubsection}{2em}{3em}

% Toc title size
\renewcommand\cfttoctitlefont{\Large\bfseries}
\renewcommand*\contentsname{Table of Contents}

% Removes bold headings from toc
%\renewcommand{\cftsecfont}{\normalfont}

% Removes bold heading page numbers from toc
\renewcommand{\cftsecpagefont}{\normalfont}

% add dots after headings
%\renewcommand{\cftsecleader}{\cftdotfill{\cftdotsep}} 


% number of section headings we want to see in toc
\setcounter{tocdepth}{2}

% Spaceing before headings in toc
\setlength{\cftbeforesecskip}{6pt}

% ------------------------------TOC FORMAT END --------------------------------%



% ------------------- START HEADER AND FOOTER ---------------------------%
\usepackage{fancyhdr}

% Helps with the n of total n pages
\usepackage{lastpage}

\pagestyle{fancy}

% Header
\lhead{Draft Component Design }
\rhead{Revision: 0}
\fancyhead[LE,CO]{VIC - Group 6}

% Removes line under the header 
\renewcommand{\headrulewidth}{0pt}
\setlength{\headsep}{.2in}

% Footer 

% Set the right side of the footer to be the page number
\fancyfoot[R]{Page \textbf{\thepage}\ of \textbf{\pageref{LastPage}}}
\fancyfoot[C]{}

% ------------------- END HEADER AND FOOTER ---------------------------%






% -------------- DOCUMENT START ---------------%
\begin{document}

% --------- TITLE PAGE START ------- %
\begin {center} 

\thispagestyle{empty}
\vspace*{5cm}

% Logo Insertion
\begin {figure}[h!]
\centering
\hspace{-10mm}\includegraphics [scale = .5, trim={.4cm 0 .8cm 0},clip] {figures/vic_logo.png}
\end {figure}

{\fontfamily{\cabinfamily}\selectfont
\Huge{Vehicle Intersection Control} }

\vspace{1 cm}
{\Large\textbf{\textsc{McMaster University}}\\}  \vspace {1cm}
{\Large Draft Component Design\\ \vspace {0.4 cm} SE 4G06}  \vspace {1cm}

{\large \textsc{Group 6} \\} \vspace{1cm}

\begin{tabular}{ l c  l}
Alex Jackson &-& 1302526\\
Jean Lucas Ferreira &-& 1152120 \\
Justin Kapinski &-& 1305257\\
Mathew Hobers &-& 1228607\\
Radhika Sharma &-& 1150430\\
Zachary Bazen &-& 1200979
\end{tabular}




\end{center}

% --------- TITLE PAGE END------- %

\pagebreak

% Inserting table of contents and table of figures 

\tableofcontents
\listoftables
\listoffigures



\pagebreak

% -----------  REVISION HISTORY START ----------- %

%\section*{Revisions}
\thispagestyle{plain}


\section{Revisions}
\begin{longtable}{| p{.2\textwidth } | p{.23\textwidth } | p{.23\textwidth } | p{.23\textwidth } |} \caption{VIC Table of Revisions}  \\

\hline 
\centering \textbf{Date} & 
\multicolumn{1}{c}{\textbf {Revision Number}} &
\multicolumn{1}{|c}{\textbf {Authors}} & 
\multicolumn{1}{|c|}{\textbf {Comments}} \\ \hline

\multirow{4}{*}{\centering January 23, 2017}  & 
\multirow{4}{*}{Revision 0}& 
{Alex Jackson \newline
Jean Lucas Ferreira \newline
Justin Kapinski\newline
Mathew Hobers\newline
Radhika Sharma\newline
Zachary Bazen}
&
 \multirow{4}{*}{N/A} \\ 
\hline 


\end{longtable}
% -----------  REVISION HISTORY END ----------- %
\pagebreak

%---------------------------- PROJECT DRIVERS ------------------------%
% heading in document

% -------------- START INTRODUCTION ---------------- %
\begin{center}
\textbf{\Large \hl{Possible Headings - Taken from suggested content. May need addtional headings}}
\end{center}

\section {Introduction}


\subsection{Document Purpose}
Insert Text Here. 

\subsubsection{System Scope}
Insert Text Here. 

\subsubsection{Document Overview and Intended Audience}
Insert Text Here. 

\subsubsection{Acronyms}
Insert Text Here. 

\subsubsection{Definitions}
Insert Text Here. 

\section{Module Guide}

\subsection{Module Overview}

% IC modules - hardware and software
\subsubsection{Intersection Controller Modules}

\begin{longtable}{ |p{0.1\textwidth }  | p{0.2\textwidth } |  p{0.3\textwidth } |  p{0.3\textwidth } |}  \hline
    
    \textbf{ID} & \textbf{Name} &  \textbf{Responsibilities} & \textbf{Secrets} \\ \hline
    
    \cellcolor{tableCell}ICM.1  & \cellcolor{tableCell}DecisionMaker & \cellcolor{tableCell}Determine order of car progression & \cellcolor{tableCell} Scheduling algorithm \\ \hline
    
    ICM.2 & VehicleDetection & Know when a car is on top of one of the intersection sensors, and the corresponding sensor & Relationship between magnetic sensor and car \\ \hline
    
    \cellcolor{tableCell}ICM.3  & \cellcolor{tableCell}Communication & \cellcolor{tableCell}Interpret receiving car signals and sending signals to a car & \cellcolor{tableCell}Communication protocol \\ \hline
    
    
        ICM.4 & IC\_Main & Control information flow of intersection controller & Manages intersection modules \\ \hline

\end{longtable}






%  Vehicle Hardware modules
\subsubsection{Vehicle Controller Hardware Modules}

\begin{longtable}{ |p{0.1\textwidth }  | p{0.2\textwidth } |  p{0.3\textwidth } |  p{0.3\textwidth } |}  \hline
    
    \textbf{ID} & \textbf{Name} &  \textbf{Responsibilities} & \textbf{Secrets} \\ \hline 
    
    
    \cellcolor{tableCell}VCM.1  & \cellcolor{tableCell}SignalConverter & \cellcolor{tableCell}Convert a software signals to a physical signal, and vice versa & \cellcolor{tableCell}How to convert signal  \\ \hline
    
    VCM.2 & SpeedConverter & Convert wheel rotation count to a speed value & Speed calculation algorithm \\ \hline
    
    \cellcolor{tableCell}VCM.3  & \cellcolor{tableCell}ServoController & \cellcolor{tableCell}Set a physical wheel angle & 
    \cellcolor{tableCell}How to convert a software value to a PWM (Pulse Width Modulation) signal \\ \hline
    
    VCM.4 & MotorSpeedController & Control PWM signal & How to convert speed into a PWM signal \\ \hline\hline
    
    \cellcolor{tableCell}VCM.5  & \cellcolor{tableCell}MotorHBridge \newline Controller & \cellcolor{tableCell}Setting H bridge gates & \cellcolor{tableCell}Which gates correspond to which action of the motor  \\ \hline
    
    
\end{longtable}


%  Vehicle Software modules
\subsubsection{Vehicle Controller Software Modules}

\begin{longtable}{ |p{0.1\textwidth }  | p{0.2\textwidth } |  p{0.3\textwidth } |  p{0.3\textwidth } |}  \hline
    
    \textbf{ID} & \textbf{Name} &  \textbf{Responsibilities} & \textbf{Secrets} \\ \hline
    
    \cellcolor{tableCell}VCM.6  &\cellcolor{tableCell}ImageProcessing &\cellcolor{tableCell}Interpret image into environment state &\cellcolor{tableCell}Image processing algorithm  \\ \hline
    
    VCM.7 & VehicleNavigaton & Control the navigation of the car & How the car navigates on the track \\ \hline
    % \cellcolor{tableCell}VCM.7.1 & \cellcolor{tableCell}VehicleNavigaton\_\newline Speed  & \cellcolor{tableCell}Ensure car speed is maintained at desired speed range & \cellcolor{tableCell}Speed control algorithm \\ \hline
    % VCM.7.2 & VehicleNavigaton\_\newline LanePositioning  & Car position with respect to lane & How to stay in lane \\ \hline
    
    \cellcolor{tableCell}VCM.8  & \cellcolor{tableCell}Communication & \cellcolor{tableCell}Interpret signal from Intersection Controller. Prepare and send signal to the Intersection Controller & \cellcolor{tableCell}Communication Protocol \\ \hline

    VCM.9 & VC\_Main & Control information flow of the car & Manage car modules \\ \hline
    
\end{longtable}

\subsection{Intersection Software Module Description}
\textbf{Design Notes} \\
<Description goes here for all the intersection software modules> \\

\begin{longtable}{p{.98\textwidth}}
\rowcolor{tableCell}\textbf{<ID and Name 1>} \\
\end{longtable}



\textbf{Behavioural Description} 

<Description goes here.> \\

\textbf{Inputs} \\
<List one after the other> \\

\textbf{Outputs} \\
<List one after the other> \\

\textbf{Initialization Description} \\
<Describe here> \\

\textbf{Derived Timing Constraints} \\
<Describe here> \\


\begin{longtable}{p{.98\textwidth}}
\rowcolor{tableCell}\textbf{<ID and Name 2>} \\
\end{longtable}

\textbf{Behavioural Description} 

<Description goes here.> \\

\textbf{Inputs} \\
<List one after the other> \\

\textbf{Outputs} \\
<List one after the other> \\

\textbf{Initialization Description} \\
<Describe here> \\

\textbf{Derived Timing Constraints} \\
<Describe here> \\


\subsection{Vehicle Software Module Description}

\textbf{Design Notes} \\
<Description goes here for all the intersection software modules> \\

\begin{longtable}{p{.98\textwidth}}
\rowcolor{tableCell}\textbf{<ID and Name 1>} \\
\end{longtable}



\textbf{Behavioural Description} 

<Description goes here.> \\

\textbf{Inputs} \\
<List one after the other> \\

\textbf{Outputs} \\
<List one after the other> \\

\textbf{Initialization Description} \\
<Describe here> \\

\textbf{Derived Timing Constraints} \\
<Describe here> \\



\begin{longtable}{p{.98\textwidth}}
\rowcolor{tableCell}\textbf{<ID and Name 2>} \\
\end{longtable}

\textbf{Behavioural Description} 

<Description goes here.> \\

\textbf{Inputs} \\
<List one after the other> \\

\textbf{Outputs} \\
<List one after the other> \\

\textbf{Initialization Description} \\
<Describe here> \\

\textbf{Derived Timing Constraints} \\
<Describe here> \\

\subsection{Vehicle Hardware Component Description}

\textbf{Design Notes\ choices} \\
<Description goes here for all the intersection software modules> \\

\begin{longtable}{p{.98\textwidth}}
\rowcolor{tableCell}\textbf{<ID and Name 1>} \\
\end{longtable}

\textbf{Inputs} \\
<List one after the other> \\

\textbf{Outputs} \\
<List one after the other> \\

\begin{longtable}{p{.98\textwidth}}
\rowcolor{tableCell}\textbf{<ID and Name 2>} \\
\end{longtable}

\textbf{Inputs} \\
<List one after the other> \\

\textbf{Outputs} \\
<List one after the other> \\



\section{Module Specifications}

There are two main components to VIC: the intersection component and the vehicle component. The following module specifications are grouped in this way. 
\subsection{Intersection Module Interface Specification}

    \begin{center}
        \hl{These may be subject to change}
    \end{center}

\begin{longtable}{| p{.35\textwidth } | p{.6\textwidth } | }\caption{ICM.1 DecisionMaker} \\\hline  
 \multicolumn{2}{|l|}{\textbf {ICM.1 DecisionMaker}}\\ \hline
 
\cellcolor{tableCell}DecisionMaker()& \cellcolor{tableCell}Constructor to initialize the scheduling algorithm \\ \hline 

getSchedule(cars[ ]) : carQueue & When function is called, it will return a queue of cars in the order which they should proceed. Expects an array of car objects when called\\ \hline 

%\cellcolor{tableCell}Constructor/Method 3 & \cellcolor{tableCell}- Description \newline - Description  
%\\ \hline


%Constructor/Method 2 & - Description 4 \newline - Description 4  \\ \hline
\end{longtable}

\begin{longtable}{| p{.35\textwidth } | p{.6\textwidth } | }\caption{ICM.2 VehicleDetection} \\\hline  
 \multicolumn{2}{|l|}{\textbf {ICM.2 VehicleDetection}}\\ \hline
 
\cellcolor{tableCell}VehicleDet()& \cellcolor{tableCell}Constructor to initialize the detection of vehicles at the intersection. \\ \hline 

getSignalsState( ) : bool[ ] & Returns the state of the sensors at the intersection when the function is called. Returns an array of boolean values signifying if the sensors have been tripped or not. \\ \hline 

%\cellcolor{tableCell}Constructor/Method 3 & \cellcolor{tableCell}- Description \newline - Description  
%\\ \hline


%Constructor/Method 2 & - Description 4 \newline - Description 4  \\ \hline
\end{longtable}

\begin{longtable}{| p{.35\textwidth } | p{.6\textwidth } | }\caption{ICM.3 Communication} \\\hline  
 \multicolumn{2}{|l|}{\textbf {ICM.3 Communication}}\\ \hline
 
\cellcolor{tableCell}RecieveRequest() : Request& \cellcolor{tableCell}Function to allow the controller recieve a request to be scheduled from the car.\\ \hline 


SendResponse(car c) : void & Function that allows the intersectrion to send a car the response to proceed through the intersection. \\ \hline




%Constructor/Method 2 & - Description 4 \newline - Description 4  \\ \hline
\end{longtable}

\begin{longtable}{| p{.35\textwidth } | p{.6\textwidth } | }\caption{ICM.4 IC\_Main} \\\hline  
 \multicolumn{2}{|l|}{\textbf {ICM.4 IC\_Main}}\\ \hline
 
\cellcolor{tableCell}Main( ) & \cellcolor{tableCell}Main Function for VIC. \\ \hline 


%\cellcolor{tableCell}Constructor/Method 3 & \cellcolor{tableCell}- Description \newline - Description  
%\\ \hline


%Constructor/Method 2 & - Description 4 \newline - Description 4  \\ \hline
\end{longtable}


\subsection{Intersection Module Internal Design}

\begin{center}
    \hl {double check whats included in the MID'S}
\end{center}


\textbf{<ID and Name 1>} \\

<Description goes here.>  \\

\textbf{Variables} 

% skip the captions --> treat it like a list or something
% Simplifies the tables for everyone
\begin{longtable}{ p{.35\textwidth }  p{.6\textwidth }} \\ 

 
\rowcolor{tableCell} <Variable 1 >& <Desciption> \\ 
<Variable 2>& <Desciption> \\

\rowcolor{tableCell}<Variable 3> & <Desciption> \\ 
<Variable 4>& <Desciption> \\

\end{longtable}

\textbf{Objects} 
% skip the captions --> treat it like a list or something
% Simplifies the tables for everyone
\begin{longtable}{ p{.35\textwidth }  p{.6\textwidth }} \\ 

 
\rowcolor{tableCell} <Object 1 >& <Desciption> \\ 
<Object 2>& <Desciption> \\

\rowcolor{tableCell}<Object 3> & <Desciption> \\ 
<Object 4>& <Desciption> \\

\end{longtable}


\textbf{Methods} 
% skip the captions --> treat it like a list or something
% Simplifies the tables for everyone
\begin{longtable}{ p{.35\textwidth }  p{.6\textwidth }} \\ 

 
\rowcolor{tableCell} <Method 1 >& <Desciption, parameters and return>\\ 
<Method 2>& <Desciption, parameters and return>\\

\rowcolor{tableCell}<Method 3> & <Desciption, parameters and return> \\ 
<Method 4>& <Desciption, parameters and return> \\

\end{longtable}






\textbf{<ID and Name 2>} \\

<Description goes here.>  \\

\textbf{Variables} 

% skip the captions --> treat it like a list or something
% Simplifies the tables for everyone
\begin{longtable}{ p{.35\textwidth }  p{.6\textwidth }} \\ 

 
\rowcolor{tableCell} <Variable 1 >& <Desciption> \\ 
<Variable 2>& <Desciption> \\

\rowcolor{tableCell}<Variable 3> & <Desciption> \\ 
<Variable 4>& <Desciption> \\

\end{longtable}

\textbf{Objects} 
% skip the captions --> treat it like a list or something
% Simplifies the tables for everyone
\begin{longtable}{ p{.35\textwidth }  p{.6\textwidth }} \\ 

 
\rowcolor{tableCell} <Object 1 >& <Desciption> \\ 
<Object 2>& <Desciption> \\

\rowcolor{tableCell}<Object 3> & <Desciption> \\ 
<Object 4>& <Desciption> \\

\end{longtable}


\textbf{Methods} 
% skip the captions --> treat it like a list or something
% Simplifies the tables for everyone
\begin{longtable}{ p{.35\textwidth }  p{.6\textwidth }} \\ 

 
\rowcolor{tableCell} <Method 1 >& <Desciption, parameters and return>\\ 
<Method 2>& <Desciption, parameters and return>\\

\rowcolor{tableCell}<Method 3> & <Desciption, parameters and return> \\ 
<Method 4>& <Desciption, parameters and return> \\

\end{longtable}








\subsection{Vehicle Module Interface Specifications}
\begin{longtable}{| p{.35\textwidth } | p{.6\textwidth } | }\caption{VCM.6 ImageProcessing} \\\hline  
 \multicolumn{2}{|l|}{\textbf {VCM.6 ImageProcessing}}\\ \hline
\cellcolor{tableCell}ImgProc( ) & \cellcolor{tableCell}Function to capture images of the track environment from a webcam and process it into information that can be analysed by software.  \\ \hline 

getImageInfo( ) : ADT & Function to relay image information when called.  \\ \hline 


\end{longtable}

\begin{longtable}{| p{.35\textwidth } | p{.6\textwidth } | }\caption{VCM.7 VehicleNavigation} \\\hline  
 \multicolumn{2}{|l|}{\textbf {VCM.7 VehicleNavigation}}\\ \hline
\cellcolor{tableCell}VehicleNav( ) & \cellcolor{tableCell}Function to signal to the vehicle if there is a change in the navigation, and if so, what changes should be made. \\ \hline 

GetCarState( ) : enum &Function to relay the car state. Will return the states as an enum. Exact states will be determined later. \\ \hline 

\cellcolor{tableCell}driveThroughIntersection( ) : void & \cellcolor{tableCell}Function to signal the car to proceed through the intersection. 
\\ \hline

\end{longtable}

\begin{longtable}{| p{.35\textwidth } | p{.6\textwidth } | }\caption{VCM.8 Communication} \\\hline  
 \multicolumn{2}{|l|}{\textbf {VCM.8 Communication}}\\ \hline
\cellcolor{tableCell}SendRequest(Request r) : void & \cellcolor{tableCell}Function to allow the car to send a request to the interection controller. \\ \hline 

Recieve Response( ) : Car &Function to allow the vehicle to revice a response to proceed from the intersection controller. \\ \hline 

\end{longtable}


\begin{longtable}{| p{.35\textwidth } | p{.6\textwidth } | }\caption{VCM.9 VC\_Main} \\\hline  
 \multicolumn{2}{|l|}{\textbf {VCM.9 VC\_Main}}\\ \hline
\cellcolor{tableCell}VC\_Main& \cellcolor{tableCell}Function to control all software aspects of the vehicle control.  \\ \hline 

\end{longtable}






\subsection{Vehicle Module Internal Design}

\begin{center}
    \hl {double check whats included in the MID'S}
\end{center}


\textbf{<ID and Name 1>} \\

<Description goes here.>  \\

\textbf{Variables} 

% skip the captions --> treat it like a list or something
% Simplifies the tables for everyone
\begin{longtable}{ p{.35\textwidth }  p{.6\textwidth }} \\ 

 
\rowcolor{tableCell} <Variable 1 >& <Desciption> \\ 
<Variable 2>& <Desciption> \\

\rowcolor{tableCell}<Variable 3> & <Desciption> \\ 
<Variable 4>& <Desciption> \\

\end{longtable}

\textbf{Objects} 
% skip the captions --> treat it like a list or something
% Simplifies the tables for everyone
\begin{longtable}{ p{.35\textwidth }  p{.6\textwidth }} \\ 

 
\rowcolor{tableCell} <Object 1 >& <Desciption> \\ 
<Object 2>& <Desciption> \\

\rowcolor{tableCell}<Object 3> & <Desciption> \\ 
<Object 4>& <Desciption> \\

\end{longtable}


\textbf{Methods} 
% skip the captions --> treat it like a list or something
% Simplifies the tables for everyone
\begin{longtable}{ p{.35\textwidth }  p{.6\textwidth }} \\ 

 
\rowcolor{tableCell} <Method 1 >& <Desciption, parameters and return>\\ 
<Method 2>& <Desciption, parameters and return>\\

\rowcolor{tableCell}<Method 3> & <Desciption, parameters and return> \\ 
<Method 4>& <Desciption, parameters and return> \\

\end{longtable}





\textbf{<ID and Name 2>} \\

<Description goes here.>  \\

\textbf{Variables} 

% skip the captions --> treat it like a list or something
% Simplifies the tables for everyone
\begin{longtable}{ p{.35\textwidth }  p{.6\textwidth }} \\ 

 
\rowcolor{tableCell} <Variable 1 >& <Desciption> \\ 
<Variable 2>& <Desciption> \\

\rowcolor{tableCell}<Variable 3> & <Desciption> \\ 
<Variable 4>& <Desciption> \\

\end{longtable}

\textbf{Objects} 
% skip the captions --> treat it like a list or something
% Simplifies the tables for everyone
\begin{longtable}{ p{.35\textwidth }  p{.6\textwidth }} \\ 

 
\rowcolor{tableCell} <Object 1 >& <Desciption> \\ 
<Object 2>& <Desciption> \\

\rowcolor{tableCell}<Object 3> & <Desciption> \\ 
<Object 4>& <Desciption> \\

\end{longtable}


\textbf{Methods} 
% skip the captions --> treat it like a list or something
% Simplifies the tables for everyone
\begin{longtable}{ p{.35\textwidth }  p{.6\textwidth }} \\ 

 
\rowcolor{tableCell} <Method 1 >& <Desciption, parameters and return>\\ 
<Method 2>& <Desciption, parameters and return>\\

\rowcolor{tableCell}<Method 3> & <Desciption, parameters and return> \\ 
<Method 4>& <Desciption, parameters and return> \\

\end{longtable}




\subsection{Vehicle Hardware Modules}

\begin{center}
    \hl{All things hardware related. We didn't do "MIS" for the hardware yet. Note the id numbers may get you confused but they follow the module guide}
\end{center}


\section{Scheduling}

\begin{center}
    \textbf{\hl{Unsure what this is"}}
\end{center}


%\section{Design Notes}
%Insert Text Here.

\section{Data Dictionary (if necessary)}
Insert Text Here.


\section{References}
Insert Text Here.


\end{document}
