

% Document uses 12 pt font
% 1 in margins
% Contains a relative path for images

\documentclass [12pt]{article}

% page geometry 
\usepackage[margin=1in]{geometry}


% ----------  PACKAGES START ------------ %

% VIC title package
\usepackage{cabin}
\usepackage[T1]{fontenc}

% default font package
\usepackage{times}

% ---------- End Font Packages -------------- %

% Title Packages
\usepackage{titlesec}
\usepackage{titletoc}

% Image Package
\usepackage{graphicx}

% Table Packages
\usepackage{longtable}
\usepackage{multirow}
\usepackage{multicol}
\usepackage{multirow}
\usepackage{array}
\renewcommand{\arraystretch}{1.4}% Spread rows out evenly in table

% List package
\usepackage{enumitem}
\setlist{nolistsep} % compresses lists

% ------------- Creates a linked Table of Contents  Start --------------- %
\usepackage{color}   %May be necessary if you want to color links
\usepackage{hyperref}
\hypersetup{
colorlinks=true, %set true if you want colored links
linktoc=all,     %set to all if you want both sections and subsections linked
linkcolor=black,  %choose some color if you want links to stand out
}
% ------------- Creates a click-able Table of Contents  End--------------- %

% ---------- PACKAGES END ------------ %

\usepackage{enumitem}
\setlist{nolistsep}


% -------- SECTION AND SUBSECTION FORMATING START -------- % 
\titleformat{\section} % Section
{\normalfont \fontsize{14}{14}\bfseries}{\thesection}{0.5em % section indentation
}{}

\titleformat{\subsection} % Subsection
{\normalfont \fontsize{12}{12} \bfseries}{\thesubsection }{0.5em}{}

\titleformat{\subsubsection} % Subsubsection
{\normalfont \fontsize{12}{12} \bfseries}{\thesubsubsection }{0.5em}{}

% -------- SECTION AND SUBSECTION FORMATING END -------- % 



% -----  IMAGE PATH START -----%
% Relative Image Path
\graphicspath {figures/}
% -----  IMAGE PATH END -----%


% -------------- DOCUMENT START ---------------%
\begin{document}

%------------------------------TOC FORMAT START --------------------------------%

\titlecontents{section}
[8pt]                                               % left margin
{}%
{\contentsmargin{2pt}                               % numbered entry format
\thecontentslabel {\enspace }  %
}
{\contentsmargin{0pt}\Large}                        % unnumbered entry format
{\titlerule*[.9pc]{ . }\contentspage} % filler-page format (e.g dots)
[ ] % below code (e.g vertical space)

\titlecontents{subsection}
[25pt]                                               % left margin
{}%
{\contentsmargin{0pt}                               % numbered entry format
\thecontentslabel\enspace\enspace%
}
{\contentsmargin{4pt}\large}                        % unnumbered entry format
{\titlerule*[.9pc]{. }\contentspage} % filler-page format (e.g dots)
[] % below code (e.g vertical space)


%------------------------------TOC FORMAT END --------------------------------%


% --------- TITLE PAGE START ------- %
\begin {center} 

\thispagestyle{empty}
\vspace*{4.5cm}



% Logo Insertion
\begin {figure}[h!]
\centering
\includegraphics [scale = .5, trim={.4cm 0 .8cm 0},clip] {figures/vic_logo.png}
\end {figure}

{\fontfamily{\cabinfamily}\selectfont
\Huge{Vehicle Intersection Control} }

\vspace{1 cm}
{\LARGE{\textsc{McMaster University}}\\}  \vspace {1cm}
{\Large Development Process and Implementation\\ \vspace {0.5 cm} SE 4G06}  \vspace {1cm}

{\large \textsc{Group 6} \\} \vspace{1cm}

{
Alex Jackson \\
Jean Lucas Ferreira \\
Justin Kapinski\\
Matthew Hobers\\
Radhika Sharma\\
Zachary Bazen}




\end{center}

% --------- TITLE PAGE END------- %

\pagebreak

% Inserting table of contents and table of figures 
\tableofcontents
\listoftables

\pagebreak

% -----------  REVISION HISTORY START ----------- %

\section{Revisions}
\begin{longtable}{| p{.23\textwidth } | p{.23\textwidth } | p{.23\textwidth } | p{.23\textwidth } |}

\hline 
\centering \textbf{Date} & 
\multicolumn{1}{c}{\textbf {Revision Number}} &
\multicolumn{1}{|c}{\textbf {Authors}} & 
\multicolumn{1}{|c|}{\textbf {Comments}} \\ \hline

\multirow{4}{*}{\centering October 22, 2016}  & 
\multirow{4}{*}{Revision 0}& 
{Alex Jackson \newline
Jean Lucas Ferreira \newline
Justin Kapinski\newline
Matthew Hobers\newline
Radhika Sharma\newline
Zachary Bazen}
&
 \multirow{4}{*}{N/A} \\ 
\hline 

\caption{VIC Table of Revisions} 
\end{longtable}
% -----------  REVISION HISTORY END ----------- %


% --------- NOTE START --------%
% Headings based on suggested content for development process & implementation found on 
% Avenue. 

% -------- Document Not Compiling ------ %
% If you can't initially compile you may need to comment out the  ---- VIC Title package ---  and
% --- Vehicle Intersection Control ---  on the title page if your version of LaTex doesn't have the appropriate font package. 

% ------------ NOTE END --------%




% ------------ DOCUMENT CONTENT HERE ------------ %
\pagebreak


% --------------- Overall Process Workflow Start ------------------------ %
\section{Overall Process Workflow}
\subsection{Project Steps and Order}
% Wassyng wants a highlevel set of steps and the order that they will completed in

% ------------ Task Table Start --------------- %
\begin{longtable}{| p{.1\textwidth } | p{.61\textwidth } | p{.23\textwidth } |}

\hline 
\centering \textbf{Step} & 
\multicolumn{1}{c}{\textbf {Task}} &
\multicolumn{1}{|c|}{\textbf {Deadline}} \\ \hline

\centering 0 & 
Develop high level project requirements and potential parts list&
\multicolumn{1}{c|}{Week of October 25} \\ \hline

\centering 1 & 
Acquire two, one tenth ($\frac {1}{10}$)  scale RC cars&
\multicolumn{1}{c|}{Week of November 1} \\ \hline

\multicolumn{1}{|c|}{\multirow{2}{*}{2}} & 
Acquire hardware to automate cars 
\begin{itemize}
	\item [{-}] micro controllers, sensor(s), camera(s) and mounting hardware
	\vspace*{-\baselineskip}
\end{itemize} &
\multicolumn{1}{c|}{\multirow{2}{*}{Week of November 1}} \\ \hline


\centering 3 & 
Develop a track plan &
\multicolumn{1}{c|}{Week of November 1} \\ \hline

\centering 4 & 
Integrate car hardware, micro-controllers and computer 
&
\multicolumn{1}{c|}{Week of November 15} \\ \hline

\centering 5 & 
 Look for open source lane following and obstacle detection &
\multicolumn{1}{c|}{Week of November 15} \\ \hline

\centering 6 & 
Lane following and obstacle detection integrated and tested &
\multicolumn{1}{c|}{Week of November 28} \\ \hline

\centering 7 & 
Intersection and Detection Algorithms Developed Concurrently  &
\multicolumn{1}{c|}{End of December} \\ \hline

\centering 8 & 
Algorithm simulations &
\multicolumn{1}{c|}{End of January} \\ \hline

\centering 9 & 
Integrate decision algorithms and intersection detection  with vehicles &
\multicolumn{1}{c|}{February} \\ \hline

\centering 10 & 
Algorithm Testing and Refinement &
\multicolumn{1}{c|}{March} \\ \hline


\centering 11 & 
Final Implementation and Documentation &
\multicolumn{1}{c|}{Beginning of April} \\ \hline

\caption{VIC Project Steps} 
\end{longtable}
% ------------ Task Table End --------------- %
 Note: Appropriate documentation will be developed as appropriate with each step.

%Very highlevel (Jean)
%(note:  sd = soft deadline)
%\begin{enumerate}
%
%\item Acquire 1 (or 2) 1/10th car models (sd: mid November)
%\item Acquire hardware (rasberry Pi, camera(s), sensors ) for each car to allow automation (sd: end of November)
%\item Integrate car models with hardware (sd: early december)
%\item Look for open-source algorithms for lane-following and apply to cars (mid december)
%\item Cars can follow lanes independently (sd: end of december)
%
%\item Algorithm considerations and design planning
%\item Test algorithms virtually (via simulations) (sd:  end january)
%\item Implement algorithm to the car software 
%\item test test test (end of february)
%
%\item freak out 
%\item ??? 
%
%\item graduate :D
%
%
%
%\end{enumerate}

% For each step provide the necessary inputs and whats the expected output. 

\subsection{Step Inputs and Outputs}
% ------------ Task Table Input/Output Start --------------- %
\begin{longtable}{| p{.1\textwidth } | p{.43\textwidth } | p{.42\textwidth } |}

\hline 
\centering \textbf{Step} & 
\multicolumn{1}{c}{\textbf {Input}} &
\multicolumn{1}{|c|}{\textbf {Output}} \\ \hline

\centering 0 & 
N/A &
High-level Design Document\\ \hline

\centering 1 & 
- &
-\\ \hline

\centering 2 & 
- &
-\\ \hline

\centering 3 & 
- &
-\\ \hline

\centering 4 & 
Hardware and Micro-controllers &
Integrated hardware/software system\\ \hline

\centering 5 & 
- &
-\\ \hline

\centering 6 & 
- &
-\\ \hline

\centering 7 & 
- &
-\\ \hline

\centering 8 & 
- &
-\\ \hline

\centering 9 & 
- &
-\\ \hline

\centering 10 & 
- &
-\\ \hline


\centering 11 & 
- &
-\\ \hline

\caption{VIC Project Steps Input and Output} 
\end{longtable}
% ------------ Task Table Input/Output End --------------- %

% What are the acceptance criteria for each of the outputs of a step
\subsection{Step Output Acceptance Criterion}

% ------------ Task Table Output Acceptance Criterion Start --------------- %
\begin{longtable}{| p{.1\textwidth } | p{.88\textwidth } |}

\hline 
\centering \textbf{Step} & 
\multicolumn{1}{c|}{\textbf {Output Acceptance Criterion}} \\ \hline

\centering 0 & 
-\\ \hline

\centering 1 & 
-\\ \hline

\centering 2 & 
-\\ \hline


\centering 3 & 
-\\ \hline

\centering 4 & 
Cars are able to be controlled from central control and micro controllers\\ \hline

\centering 5 & 
-\\ \hline

\centering 6 & 
-\\ \hline

\centering 7 & 
-\\ \hline

\centering 8 & 
-\\ \hline

\centering 9 & 
-\\ \hline

\centering 10 & 
-\\ \hline


\centering 11 & 
-\\ \hline

\caption{VIC Project Steps Output Acceptance Criterion} 
\end{longtable}
% ------------ Task Table Output Acceptance Criterion End --------------- %

% --------------- Overall Process Workflow End ------------------------ %




% ------------------- Step Completion Start ---------------------------- %
% Provide some detains on how each step should be done.
\section{Step Completion Information}

% What tools are you going to use? What versions?
\subsection{Tools and Versions}
TBD

% Are there any special instructions on settings or how to use the tools? What
% information to put in version control? 
\subsection{Tool Setting and Use}
TBD


% Any standards you should follow? (e.g. coding standards)
\subsection{Standards}
Volere and IEEE for software requirements specification \\
Coding standards and conventions, for the programming language at hand.\\

% Who should perform each step. 
\subsection{Work Assignments}

Ideally we should create two subgroups (HW and SW), but we would still
discuss both aspects as a whole group.\\ When it comes to implementation subgroups might be more efficient. \\

% ------------ Assignment Table Start --------------- %
\begin{longtable}{| p{.1\textwidth } | p{.71\textwidth } | p{.23\textwidth } |}

\hline 
\centering \textbf{Step} & 
\multicolumn{1}{c}{\textbf {Task}} &
\multicolumn{1}{|c|}{\textbf {Assignment}} \\ \hline

\centering 0 & 
Develop high level project requirements and potential parts list&
\multicolumn{1}{c|}{VIC} \\ \hline

\centering 1 & 
Acquire two, one tenth ($\frac {1}{10}$)  scale RC cars&
\multicolumn{1}{c|}{Name} \\ \hline

\multicolumn{1}{|c|}{\multirow{2}{*}{2}} & 
Acquire hardware to automate cars 
\begin{itemize}
	\item [{-}] micro controllers, sensor(s), camera(s) and mounting hardware
	\vspace*{-\baselineskip}
\end{itemize} &
\multicolumn{1}{c|}{\multirow{2}{*}{Name}} \\ \hline


\centering 3 & 
Develop a track plan &
\multicolumn{1}{c|}{VIC} \\ \hline

\centering 4 & 
Integrate car hardware, micro-controllers and computer 
&
\multicolumn{1}{c|}{HD} \\ \hline

\centering 5 & 
 Look for open source lane following and obstacle detection &
\multicolumn{1}{c|}{HD} \\ \hline

\centering 6 & 
Lane following and obstacle detection integrated and tested &
\multicolumn{1}{c|}{SE} \\ \hline

\centering 7 & 
Intersection and Detection Algorithms Developed Concurrently  &
\multicolumn{1}{c|}{VIC} \\ \hline

\centering 8 & 
Algorithm simulations &
\multicolumn{1}{c|}{SE} \\ \hline

\centering 9 & 
Integrate decision algorithms and intersection detection  with vehicles &
\multicolumn{1}{c|}{VIC} \\ \hline

\centering 10 & 
Algorithm Testing and Refinement &
\multicolumn{1}{c|}{SE} \\ \hline


\centering 11 & 
Final Implementation and Documentation &
\multicolumn{1}{c|}{VIC} \\ \hline

\caption{VIC Project Assignments} 
\end{longtable}


% ------------ Assignment Table End --------------- %
\noindent{\textbf{Table Key}}\\
SE: Software Team (Name, Name Name)\\
HD: Hardware Team (Name, Name, Name) \\
VIC: Whole VIC Team\\

\noindent{Note:} These assignments are tentative and subject to change as project advances. 
% ------------------- Step Completion End ---------------------------- %




% -------------------- Version Control Information Start ---------------------- % 

% How are you using version control?
\section{Version Control Information}
The version control of choice for this project is GitHub. Two repositories will likely be required: one for documentation and miscellaneous information and another for source code, libraries and dependencies. 
%GitHub
%Ideally \textbf{two} repos: 
%\begin{itemize} 
%\item  documentation and miscellaneous stuff
%\item  source code, libraries, and dependencies.
%\end{itemize}
% --------------------- Version Control Information End ---------------------- % 





% -------------------- Project Evolution Start ------------------------- %
% How are you going to deal with changes to development artifacts
\section{Project Evolution}

% What bug tracking/change request tool?
\subsection{Bug and Change Tracking}
Any issues with the project (i.e. bugs) will  be posted on GitHub via the Issues panel. When a issue is posted, the appropriate members will take responsibility to fixing the bug. Once fixed, the issue will be closed. Appropriate members will include: developers, software team, hardware team or both. 

% How do you document change requests/bugs?
\subsection{Project Change Documentation}
VIC will log project changes through GitHub version control logs, personal log books and VIC documentation. 

% How do you classify Changes
\subsection{Project Change Classification}
VIC will classify changes in the following ways: Global change, Software change and Hardware change. \\\\
A global change constitutes a change that affects both hardware and software aspects of the system. This type of change would fundamentally alter the functionality of the system. \\ \\
A software change only affects the software aspects of the system. \\\\
A hardware change only affects the hardware aspects of the system.   




%\pagebreak
% How do you disposition them (decide what to do, verify that they have been
%completed, etc)
\subsection{Making Project Change Decisions} 
% ------------ Making Project Change Decision Table Start --------------- %
\begin{longtable}{| p{.15\textwidth } | p{.22\textwidth } | p{.25\textwidth } | p{.29\textwidth } |}
\hline
\centering \textbf{Change Type} & 
\multicolumn{1}{c|}{\textbf {Change Severity}} &
\multicolumn{1}{c}{\textbf {Decision Assignment}} & 
\multicolumn{1}{|c|}{\textbf {Change Documentation}}  \\ \hline

\multicolumn{1}{|c|}{\multirow{3}{*}{Global}} & 
\multicolumn{1}{c}{Software} &
\multicolumn{1}{|c|}{VIC Team} &
Logged in VIC documents	\\\cline{2-4}

 & \multicolumn{1}{c}{Hardware} & 
 \multicolumn{1}{|c|}{VIC Team} &
Logged in VIC documents\\ \cline{2-4}
  
   & \multicolumn{1}{c}{Other} & 
 \multicolumn{1}{|c|}{VIC Team} &
Logged in VIC documents\\ \hline
 
 
\multicolumn{1}{|c|}{\multirow{2}{*}{Software}} & 
\multicolumn{1}{c}{Local} &
\multicolumn{1}{|c|}{Developer}&
GitHub Commit Log \\\cline{2-4}
 
 & 
\multicolumn{1}{c}{API Change} &
\multicolumn{1}{|c|}{Software Team}&
Logged in VIC documents \\ \hline


\multicolumn{1}{|c|}{\multirow{2}{*}{Hardware}} & 
\multicolumn{1}{c}{Local} &
\multicolumn{1}{|c|}{Developer}&
Log Books \\\cline{2-4}
 
 & 
\multicolumn{1}{c}{Interface Change} &
\multicolumn{1}{|c|}{Hardware Team}&
Logged in VIC documents \\ \hline


\end{longtable}
% ------------ Making Project Change Decision Table End --------------- %

% -------------------- Project Evolution End ------------------------- %

\end{document}
% -------------- DOCUMENT END ---------------%






